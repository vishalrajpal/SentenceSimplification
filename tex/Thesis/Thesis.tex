\documentclass[11pt]{article}

\begin{document}

\title{Learning to Solve Arithmetic Word Problems Using Sentence Simplification}
\author{
	Vishal Rajpal\\
	Northeastern University\\
	rajpal.vi@husky.neu.edu\\
}
\date{}
\maketitle

\begin{abstract}
In order to respond to an arithmetic word problem correctly one needs to understand the question to an extent which allows determining the constraints. These are mostly semantic and are imposed by the question on its answer. Constraints from individual sentences suggest a mathematical operation and when the operators from these sentences are used collectively the answer can be derived. To extract the constraints efficiently, a concept of Syntactic Pattern is introduced which is generated by parsing the sentence using a dependency parser. It encapsulates all the relevant information in a sentence including the subject, verb and object with its quantity. Another important method for this thesis which relies on syntactic patterns is Sentence Simplification. The idea is to have a subject, verb, an object and other necessary parts of speech in the sentence so that it suggests a single operation. Based on the identified patterns a sentence may be simplified to multiple sentences. This would make the classification process easier since the sentences are less complex. To this end, it would be the classifier's job to classify the sentence to a mathematical operator. The identified operators for individual sentences are used to build a mathematical equation for the entire word problem. The results based on 3 datasets are reported and seem promising as compared to the existing systems.
\end{abstract}

\section{Introduction}

\end{document}